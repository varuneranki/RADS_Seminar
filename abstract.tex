\noindent

Distant Supervision Relation Extraction (DSRE), supports users by providing automatically annotated data by using supervised and semi-supervised learning methods. A novel Hybrid Graph Model aids DSRE, by providing a framework for unsupervised learning methods. Limitations of other DSRE approaches are taken into consideration, and thereby, extended for different domains by incorporating heterogeneous background information. Previously, semi-supervised approaches incorporated, a specific type of background information depending, on a specific domain which achieved average results. As a downside, they cannot be extended to other domains and need customization. This novel approach generalizes the relation extraction problem. This approach outperforms, all other commonly used DSRE baseline approaches. It can be used with all models as it is flexible in embedding different kinds of domain information for relation extraction. Noisy data is gracefully handled in most cases using the built-in attention mechanism, that other approaches lack of. In \txti{"A Hybrid Graph Model for Distant Supervision Relation Extraction"}, authors \txti{Duan} et al., explores the benefits of using an Graph Convolution Networks with attention mechanism. Noisy data problem still persists and further evaluation of this method will yield, much better results in the future.
\noindent
%Recommender systems,aid users in decision-making tasks.They're very essential these days,across multiple domains, as the users are provided with plethora of choices most of the time. As such,these Recommender systems utilize knowledge on the domain, often combined with users own previous choices to better predict the items he/she might like in the future. One such way of deriving the knowledge of the domain is through leveraging the Linked Data. The Linked Data,or Knowledge Graphs are then extracted for relevant features that might better fit the scenario of prediction at hand. The feature selection,as of today is mostly done by techniques that mostly rely on the data and its distribution in the dataset. In \txti{"Using Ontology-based Data Summarization to develop Semantics-aware Recommender Systems"}, authors \txti{Tommaso} et.al., explore an alternative approach that is deeply ingrained in the semantics of the data rather than the data itself. The consequent benefits of this approach are explained along with the results when compared with most of the other state-of-the-art techniques. The better results make a case for the further evaluation and expansion of the technique and could act as an initiating point in better,semantic infused solutions for the Recommendation problem.


Write at the last
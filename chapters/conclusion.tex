\chapter{Conclusion and Future work}
\label{ch:conclusion}

To conclude, authors \txti{S. Duan} et al.\cite{duan2019hybrid}, in the work \txti{“A Hybrid Graph Model for Distant Supervision Relation Extraction”} propose a novel and different approach to incorporate heterogeneous information for DSRE. For this model, the core principles of semi-supervised learning models borrowed and their limitations we eliminated to a great extent. The vector representation of data enable to hand-pick specific features, by a simple aggregated weight mechanism at each level. Memory over-head to store large graphs is completely eliminated. A real-world data-set was chosen, and this approach achieves better results. Authors claim that the approach works best for known good data, and manages to find a long-tail relation for unseen data. Claims were also made that, this approach is applicable to any type of unstructured data in all application domains. The proposed technique should be explored further over other application domains to prove the claims. The future work comprises of reduction of noisy data. Data-sets with particularly sparse graphs should be chosen, to find a specific pattern, to alleviate the noisy data completely. The authors intend to solve this problem by devising an efficient method, along with generalizing this approach to large unlabeled text for learning more confident information. 
